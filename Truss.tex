
% This LaTeX was auto-generated from MATLAB code.
% To make changes, update the MATLAB code and republish this document.

\documentclass{article}
\usepackage{graphicx}
\usepackage{color}

\sloppy
\definecolor{lightgray}{gray}{0.5}
\setlength{\parindent}{0pt}

\begin{document}

    
    
\subsection*{Contents}

\begin{itemize}
\setlength{\itemsep}{-1ex}
   \item construct the load table
   \item construct the joint table
   \item construct the member table
   \item check for wrong variable names
   \item check for wrong variable types
   \item check for wrong categories
   \item check for \begin{verbatim}undefined\end{verbatim} categories
   \item check for wrong variable names
   \item check for wrong variable types
   \item check for wrong variable names
   \item check for wrong variable types
   \item check the input arguments
   \item add the loads, joints, and or/members to the truss
   \item check the input arguments
   \item case for non-scalar trusses
   \item convert to cell arrays of symbolic arrays
   \item check the input arguments
   \item case for non-scalar trusses
   \item convert to cell arrays of symbolic character vectors
   \item check the input arguments
   \item case for non-scalar trusses
   \item convert to string arrays
   \item check the input arguments
   \item case for non-scalar trusses
   \item set the E property
   \item check the input arguments
   \item case for non-scalar trusses
   \item set the E property
   \item check the input arguments
   \item case for non-scalar trusses
   \item shift the loads, joints, and members on the truss
   \item check the input arguments
   \item case for non-scalar trusses
   \item convert the table variables to cell arrays of symbolic arrays
   \item compute these useful values
   \item inspect the magnitudes and distances
   \item inspect the loads
   \item inspect the joints
   \item inspect the members
   \item compute the global stiffness matrix data
   \item compute the global stiffness matrix
   \item compute the global displacement and force vectors
   \item solve for the unknown variables
   \item check the input arguments
   \item determine if the truss is stable
   \item check the input arguments
   \item determine if the truss is statically determinate
\end{itemize}
\begin{verbatim}
classdef Truss
  % ==
  % -----------------
  % - the truss class
  % -----------------
  % ==
  properties % truss properties
    load;
    joint;
    member;
  end
  % ==
  properties (Constant) % table variable and category names
    load_varnames = {'class' 'magnitude' 'distance'}.';
    joint_varnames = {'distance'};
    member_varnames = {'start_distance' 'end_distance' 'E' 'A'}.';
    class_categories = {'reaction' 'concentrated'}.';
  end
  % ==
  methods % truss constructor
    % ==
    function t = Truss
\end{verbatim}
\begin{verbatim}
      % -----------------------
      % - the truss constructor
      % -----------------------
\end{verbatim}


\subsection*{construct the load table}

\begin{verbatim}
      persistent tbl1;
      if ~istable(tbl1)
        Args = {"concentrated" Truss.class_categories 'Protected' true};
        tbl1.class = categorical(Args{:});
        tbl1.magnitude = "";
        tbl1.distance = "";
        tbl1 = struct2table(tbl1);
        tbl1(:,:) = [];
      end
      t.load = tbl1;
\end{verbatim}


\subsection*{construct the joint table}

\begin{verbatim}
      persistent tbl2;
      if ~istable(tbl2)
        tbl2.distance = "";
        tbl2 = struct2table(tbl2);
        tbl2(:,:) = [];
      end
      t.joint = tbl2;
\end{verbatim}


\subsection*{construct the member table}

\begin{verbatim}
      persistent tbl3;
      if ~istable(tbl3)
        tbl3.start_distance = "";
        tbl3.end_distance = "";
        tbl3.E = "";
        tbl3.A = "";
        tbl3 = struct2table(tbl3);
        tbl3(:,:) = [];
      end
      t.member = tbl3;
\end{verbatim}
\begin{verbatim}
    end
    % ==
    function t = set.load(t, rhs)
\end{verbatim}
\begin{verbatim}
      % ---------------------
      % - sets the load table
      % ---------------------
\end{verbatim}


\subsection*{check for wrong variable names}

\begin{verbatim}
      varnames = Truss.load_varnames.';
      if ~istable(rhs) || ~isperm(rhs.Properties.VariableNames, varnames)
        str = stack('''load'' property must be a table', ...
                    'with the following variable names:', ...
                    '----------------------------------', ...
                    '1.) ''class''', ...
                    '2.) ''magnitude''', ...
                    '3.) ''distance''');
        error(str);
      end
\end{verbatim}


\subsection*{check for wrong variable types}

\begin{verbatim}
      if ~iscatcol(rhs.class)
        error('''class'' variable must be a categorical column vector');
      end
      if (~iscellcol(rhs.magnitude) && ~isStringCol(rhs.magnitude)) || ...
         (~iscellcol(rhs.distance) && ~isStringCol(rhs.distance))
        str = stack('''magnitude'' and ''distance'' variables', ...
                    'must be cell or string column vectors');
        error(str);
      end
\end{verbatim}


\subsection*{check for wrong categories}

\begin{verbatim}
      if ~isprotected(rhs.class) || isordinal(rhs.class)
        str = stack('''class'' variable', ...
                    'must be a protected and unordinal', ...
                    'categorical array');
        error(str);
      end
      if ~isperm(Truss.class_categories, categories(rhs.class))
        str = stack('''class'' variable must have', ...
                    'the following categories:', ...
                    '-------------------------', ...
                    '1.) ''reaction''', ...
                    '2.) ''concentrated''');
        error(str);
      end
\end{verbatim}


\subsection*{check for \begin{verbatim}undefined\end{verbatim} categories}

\begin{verbatim}
      if any(isundefined(rhs.class))
        error('''class'' variable must not have ''<undefined>'' values');
      end
      t.load = rhs;
\end{verbatim}
\begin{verbatim}
    end
    % ==
    function t = set.joint(t, rhs)
\end{verbatim}
\begin{verbatim}
      % ----------------------
      % - sets the joint table
      % ----------------------
\end{verbatim}


\subsection*{check for wrong variable names}

\begin{verbatim}
      varnames = Truss.joint_varnames.';
      if ~istable(rhs) || ~isperm(rhs.Properties.VariableNames, varnames)
        str = stack('''joint'' property must be a table', ...
                    'with the following variable name:', ...
                    '---------------------------------', ...
                    '1.) ''distance''');
        error(str);
      end
\end{verbatim}


\subsection*{check for wrong variable types}

\begin{verbatim}
      if ~iscellcol(rhs.distance) && ~isStringCol(rhs.distance)
        str = stack('''joint'' variable must be', ...
                    'a cell or string column vector');
        error(str);
      end
      t.joint = rhs;
\end{verbatim}
\begin{verbatim}
    end
    % ==
    function t = set.member(t, rhs)
\end{verbatim}
\begin{verbatim}
      % -----------------------
      % - sets the member table
      % -----------------------
\end{verbatim}


\subsection*{check for wrong variable names}

\begin{verbatim}
      varnames = Truss.member_varnames.';
      if ~istable(rhs) || ~isperm(rhs.Properties.VariableNames, varnames)
        str = stack('''member'' property must be a table', ...
                    'with the following variable names:', ...
                    '----------------------------------', ...
                    '1.) ''start_distance''', ...
                    '2.) ''end_distance''', ...
                    '3.) ''E''', ...
                    '4.) ''A''');
        error(str);
      end
\end{verbatim}


\subsection*{check for wrong variable types}

\begin{verbatim}
      if (~iscellcol(rhs.start_distance) && ...
          ~isStringCol(rhs.start_distance)) || ...
         (~iscellcol(rhs.end_distance) && ...
          ~isStringCol(rhs.end_distance)) || ...
         (~iscellcol(rhs.E) && ~isStringCol(rhs.E)) || ...
         (~iscellcol(rhs.A) && ~isStringCol(rhs.A))
        str = stack('''member'' variables must be', ...
                    'cell or string column vectors');
        error(str);
      end
      t.member = rhs;
\end{verbatim}
\begin{verbatim}
    end
    % ==
  end
  % ==
  methods % truss calculation member methods
    % ==
    function t = add(t, what, varargin)
\end{verbatim}
\begin{verbatim}
      % -----------------------------------
      % - add loads, joints, and/or members
      %   to the truss
      % -----------------------------------
\end{verbatim}


\subsection*{check the input arguments}

\begin{par}
check the argument classes
\end{par} \vspace{1em}
\begin{verbatim}
      arguments
        t;
        what ...
        {mustBeTextScalar, ...
         mustBeMemberi(what, ["reaction" "concentrated" ...
                              "joint" "member"])};
      end
      arguments (Repeating)
        varargin;
      end
      % check the adding argument
      what = lower(what);
\end{verbatim}


\subsection*{add the loads, joints, and or/members to the truss}

\begin{verbatim}
      switch what
        case {"reaction" "concentrated"}
          % check the table arguments
          if length(varargin) ~= 3
            str = stack('there must be 3 table arguments passed', ...
                        'when adding loads to the truss');
            error(str);
          end
          % add the load to the truss
          if isempty(varargin{1})
            rowname = ['load' num2str(height(t.load)+1)];
          else
            rowname = varargin{1};
          end
          class = what;
          magnitude = array2symstr(varargin{2});
          distance = array2symstr(varargin{3});
          t.load(end+1,:) = {class magnitude distance};
          empty_rownames = isempty(t.load.Properties.RowNames);
          one_load = height(t.load) == 1;
          if empty_rownames && one_load
            t.load.Properties.RowNames(end+1) = {rowname};
          elseif ~empty_rownames
            t.load.Properties.RowNames{end} = rowname;
          end
        case "joint"
          % check the table arguments
          if length(varargin) ~= 2
            str = stack('there must 2 table arguments passed', ...
                        'when adding joints to the truss');
            error(str);
          end
          % add the joint to the truss
          if isempty(varargin{1})
            rowname = num2str(height(t.joint)+1);
          else
            rowname = varargin{1};
          end
          distance = array2symstr(varargin{2});
          t.joint(end+1,:) = {distance};
          empty_rownames = isempty(t.joint.Properties.RowNames);
          one_joint = height(t.joint) == 1;
          if empty_rownames && one_joint
            t.joint.Properties.RowNames(end+1) = {rowname};
          elseif ~empty_rownames
            t.joint.Properties.RowNames{end} = rowname;
          end
        case "member"
          % check the table arguments
          if (length(varargin) < 3) || (length(varargin) > 5)
            str = stack('there must be at least 3,', ...
                        'but at most 5 table arguments passed', ...
                        'when adding members to the truss');
            error(str);
          end
          % add the member to the truss
          if isempty(varargin{1})
            rowname = ['member' num2str(height(t.member)+1)];
          else
            rowname = varargin{1};
          end
          start_distance = array2symstr(varargin{2});
          end_distance = array2symstr(varargin{3});
          if length(varargin) == 3
            E = 'E';
            A = 'A';
          elseif length(varargin) == 4
            E = array2symstr(varargin{4});
            A = 'A';
          else
            if isempty(varargin{4})
              E = 'E';
            else
              E = array2symstr(varargin{4});
            end
            A = array2symstr(varargin{5});
          end
          t.member(end+1,:) = {start_distance end_distance E A};
          empty_rownames = isempty(t.member.Properties.RowNames);
          one_member = height(t.member) == 1;
          if empty_rownames && one_member
            t.member.Properties.RowNames(end+1) = {rowname};
          elseif ~empty_rownames
            t.member.Properties.RowNames{end} = rowname;
          end
      end
\end{verbatim}
\begin{verbatim}
    end
    % ==
    function t = convert2cellsym(t, options)
\end{verbatim}
\begin{verbatim}
      % ----------------------------------
      % - converts the variables of the
      %   load, joint, and member tables
      %   to cell arrays of symbolc arrays
      % ----------------------------------
\end{verbatim}


\subsection*{check the input arguments}

\begin{par}
check the argument classes
\end{par} \vspace{1em}
\begin{verbatim}
      arguments
        t;
        options.Mode ...
        {mustBeTextScalar, ...
         mustBeMemberi(options.Mode, ["ignore" "nan"])} = "ignore";
        options.Prop ...
        {mustBeText, ...
         mustBeMemberi(options.Prop, ["load" "joint" "member"])};
      end
      % check the conversion mode
      Mode = lower(options.Mode);
      % check the truss property
      if isfield(options, 'Prop')
        Prop = unique(string(lower(options.Prop)), 'stable');
      else
        Prop = ["load" "joint" "member"];
      end
\end{verbatim}


\subsection*{case for non-scalar trusses}

\begin{verbatim}
      if isempty(t)
        return;
      elseif ~isscalar(t)
        for k = 1:numel(t)
          t(k) = t(k).convert2cellsym('Mode', Mode, 'Prop', Prop);
        end
        return;
      end
\end{verbatim}


\subsection*{convert to cell arrays of symbolic arrays}

\begin{verbatim}
      switch Mode
        case "ignore"
          Args = {'ErrorHandler' @ignore_args 'UniformOutput' false};
        case "nan"
          Args = {'ErrorHandler' @nan_args 'UniformOutput' false};
      end
      for k = Prop
        vars = string(t.(k).Properties.VariableNames);
        if k == "load"
          vars = vars(2:end);
        end
        for p = vars
          t.(k).(p) = cellfun(@array2sym, t.(k).(p), Args{:});
        end
      end
\end{verbatim}
\begin{verbatim}
    end
    % ==
    function t = convert2cellsymstr(t, options)
\end{verbatim}
\begin{verbatim}
      % -------------------------------------------
      % - converts the variables of the
      %   load, joint, and member tables to
      %   cell arrays of symbolic character vectors
      % -------------------------------------------
\end{verbatim}


\subsection*{check the input arguments}

\begin{par}
check the argument classes
\end{par} \vspace{1em}
\begin{verbatim}
      arguments
        t;
        options.Mode ...
        {mustBeTextScalar, ...
         mustBeMemberi(options.Mode, ["ignore" "nan"])} = "ignore";
        options.Prop ...
        {mustBeText, ...
         mustBeMemberi(options.Prop, ["load" "joint" "member"])};
      end
      % check the conversion mode
      Mode = lower(options.Mode);
      % check the truss property
      if isfield(options, 'Prop')
        Prop = unique(string(lower(options.Prop)), 'stable');
      else
        Prop = ["load" "joint" "member"];
      end
\end{verbatim}


\subsection*{case for non-scalar trusses}

\begin{verbatim}
      if isempty(t)
        return;
      elseif ~isscalar(t)
        for k = 1:numel(t)
          t(k) = t(k).convert2cellstr('Mode', Mode, 'Prop', Prop);
        end
        return;
      end
\end{verbatim}


\subsection*{convert to cell arrays of symbolic character vectors}

\begin{verbatim}
      switch Mode
        case "ignore"
          Args = {'ErrorHandler' @ignore_args 'UniformOutput' false};
        case "nan"
          Args = {'ErrorHandler' @nan_args 'UniformOutput' false};
      end
      for k = Prop
        vars = string(t.(k).Properties.VariableNames);
        if k == "load"
          vars = vars(2:end);
        end
        for p = vars
          t.(k).(p) = cellfun(@array2symstr, t.(k).(p), Args{:});
          symbolics = cellfun(@issym, t.(k).(p));
          t.(k).(p)(symbolics) = {'NaN'};
        end
      end
\end{verbatim}
\begin{verbatim}
    end
    % ==
    function t = convert2string(t, options)
\end{verbatim}
\begin{verbatim}
      % --------------------------------
      % - converts the variables of the
      %   load, joint, and member tables
      %   to string arrays
      % --------------------------------
\end{verbatim}


\subsection*{check the input arguments}

\begin{par}
check the argument classes
\end{par} \vspace{1em}
\begin{verbatim}
      arguments
        t;
        options.Mode ...
        {mustBeTextScalar, ...
         mustBeMemberi(options.Mode, ["ignore" "nan"])} = "ignore";
        options.Prop ...
        {mustBeText, ...
         mustBeMemberi(options.Prop, ["load" "joint" "member"])};
      end
      % check the conversion mode
      Mode = lower(options.Mode);
      % check the truss property
      if isfield(options, 'Prop')
        Prop = unique(string(lower(options.Prop)), 'stable');
      else
        Prop = ["load" "joint" "member"];
      end
\end{verbatim}


\subsection*{case for non-scalar trusses}

\begin{verbatim}
      if isempty(t)
        return;
      elseif ~isscalar(t)
        for k = 1:numel(t)
          t(k) = t(k).convert2string('Mode', Mode, 'Prop', Prop);
        end
        return;
      end
\end{verbatim}


\subsection*{convert to string arrays}

\begin{verbatim}
      switch Mode
        case "ignore"
          Args = {'ErrorHandler' @ignore_args 'UniformOutput' false};
        case "nan"
          Args = {'ErrorHandler' @nan_args 'UniformOutput' false};
      end
      for k = Prop
        vars = string(t.(k).Properties.VariableNames);
        if k == "load"
          vars = vars(2:end);
        end
        for p = vars
          t.(k).(p) = cellfun(@array2symstr, t.(k).(p), Args{:});
          chars = cellfun(@ischar, t.(k).(p));
          symbolics = cellfun(@issym, t.(k).(p));
          t.(k).(p)(~chars & ~symbolics) = {missing};
          t.(k).(p) = string(t.(k).(p));
          t.(k).(p)(symbolics) = "NaN";
        end
      end
\end{verbatim}
\begin{verbatim}
    end
    % ==
    function t = setE(t, Enew)
\end{verbatim}
\begin{verbatim}
      % ------------------------------------
      % - sets the E property of
      %   all members to the value of 'Enew'
      % ------------------------------------
\end{verbatim}


\subsection*{check the input arguments}

\begin{par}
check the argument classes
\end{par} \vspace{1em}
\begin{verbatim}
      arguments
        t;
        Enew sym;
      end
      % check the argument dimensions
      if ~isScalar(Enew) && ~isVector(Enew, 'Len', length(t.member.E))
        str = stack('''Enew'' must be:', ...
                    '---------------', ...
                    '1.) a scalar', ...
                    '2.) a vector with a', ...
                    '    length equal to the', ...
                    '    number of truss members');
        error(str);
      end
\end{verbatim}


\subsection*{case for non-scalar trusses}

\begin{verbatim}
      if isempty(t)
        return;
      elseif ~isscalar(t)
        for k = 1:numel(t)
          t(k) = t(k).setE(Enew);
        end
        return;
      end
\end{verbatim}


\subsection*{set the E property}

\begin{verbatim}
      if iscellsym(t.member.E)
        t.member.E(:) = array2cellsym(Enew);
      elseif iscell(t.member.E)
        t.member.E(:) = array2cellsymstr(Enew);
      else
        t.member.E(:) = array2string(Enew);
      end
\end{verbatim}
\begin{verbatim}
    end
    % ==
    function t = setA(t, Anew)
\end{verbatim}
\begin{verbatim}
      % ------------------------------------
      % - sets the A property of
      %   all members to the value of 'Anew'
      % ------------------------------------
\end{verbatim}


\subsection*{check the input arguments}

\begin{par}
check the argument classes
\end{par} \vspace{1em}
\begin{verbatim}
      arguments
        t;
        Anew sym;
      end
      % check the argument dimensions
      if ~isScalar(Anew) && ~isVector(Anew, 'Len', length(t.member.E))
        str = stack('''Anew'' must be:', ...
                    '---------------', ...
                    '1.) a scalar', ...
                    '2.) a vector with a', ...
                    '    length equal to the', ...
                    '    number of truss members');
        error(str);
      end
\end{verbatim}


\subsection*{case for non-scalar trusses}

\begin{verbatim}
      if isempty(t)
        return;
      elseif ~isscalar(t)
        for k = 1:numel(t)
          t(k) = t(k).setA(Anew);
        end
        return;
      end
\end{verbatim}


\subsection*{set the E property}

\begin{verbatim}
      if iscellsym(t.member.A)
        t.member.A(:) = array2cellsym(Anew);
      elseif iscell(t.member.A)
        t.member.A(:) = array2cellsymstr(Anew);
      else
        t.member.A(:) = array2string(Anew);
      end
\end{verbatim}
\begin{verbatim}
    end
    % ==
    function t = shift_joints(t, shift)
\end{verbatim}
\begin{verbatim}
      % -------------------------------
      % - shift the joints on the truss
      % -------------------------------
\end{verbatim}


\subsection*{check the input arguments}

\begin{par}
check the argument classes
\end{par} \vspace{1em}
\begin{verbatim}
      arguments
        t;
        shift sym;
      end
      % check the argument dimensions
      shift = formula(shift);
      if ~isvector(shift)
        error('''shift'' must be a vector');
      end
\end{verbatim}


\subsection*{case for non-scalar trusses}

\begin{verbatim}
      if isempty(t)
        return;
      elseif ~isscalar(t)
        for k = 1:numel(t)
          t(k) = t(k).shift_joints(shift);
        end
        return;
      end
\end{verbatim}


\subsection*{shift the loads, joints, and members on the truss}

\begin{verbatim}
      for k = ["load" "joint" "member"]
        for p = 1:height(t.(k))
          try
            if ismember(k, ["load" "joint"])
              % shift the loads and joints
              new_distance = array2sym(t.(k).distance(p))+shift;
              if ischar(t.(k).distance{p})
                new_distance = char(new_distance);
              end
              t.(k).distance{p} = new_distance;
            else
              % shift the starting side of the members
              new_distance = array2sym(t.(k).start_distance(p))+shift;
              if ischar(t.(k).start_distance{p})
                new_distance = char(new_distance);
              end
              t.(k).start_distance{p} = new_distance;
              % shift the ending side of the members
              new_distance = array2sym(t.(k).end_distance(p))+shift;
              if ischar(t.(k).end_distance{p})
                new_distance = char(new_distance);
              end
              t.(k).end_distance{p} = new_distance;
            end
          catch
          end
        end
      end
\end{verbatim}
\begin{verbatim}
    end
    % ==
    function [us ua ms ma ls la] = solve(t, options)
\end{verbatim}
\begin{verbatim}
      % ------------------------------------------------
      % - solves the truss for the following information
      %   by using the stiffness matrix method:
      % - 1.) the joint displacements
      %   2.) the forces and moments of each member
      %   3.) the forces and moments of each load
      % ------------------------------------------------
\end{verbatim}


\subsection*{check the input arguments}

\begin{par}
check the argument classes
\end{par} \vspace{1em}
\begin{verbatim}
      arguments
        t;
        options.Mode ...
        {mustBeText, ...
         mustBeMemberi(options.Mode, ["simplify" ...
                                      "factor" ...
                                      "factor full" ...
                                      "factor complex" ...
                                      "factor real" ...
                                      "simplify fraction" ...
                                      "simplify fraction expand"])} = ...
                                      "simplify";
        options.Reference ...
        {mustBeA(options.Reference, ["numeric" "sym" ...
                                     "cell" "char" ...
                                     "string"])} = "default";
      end
      % check the simplification mode
      Mode = string(lower(options.Mode));
      if ~isStringArray(Mode, 'ArrayDim', size(t)) && ...
         ~isStringScalar(Mode) && ~isempty(t)
        str = stack('the simplification mode must be', ...
                    'a text array with the same size as the beam');
        error(str);
      end
      % check the reference points
      Reference = options.Reference;
      if isnumvector(Reference) || issymvector(Reference) || ...
         isequal(Reference, "default")
        Reference = {Reference};
      end
      uniform = {'UniformOutput' false};
      numvectors = cellfun(@isnumvector, Reference);
      symvectors = cellfun(@issymvector, Reference);
      symfuns = cellfun(@issymfun, Reference);
      func = @(arg) isequal(arg, "default");
      using_default = cellfun(func, Reference);
      Reference(symfuns) = cellfun(@formula, ...
                                   Reference(symfuns), uniform{:});
      if ~iscellarray(Reference, 'CellDim', size(t)) && ...
         ~iscellscalar(Reference) && ~isempty(t)
        str = stack('the size of the cell array', ...
                    'for the reference points must', ...
                    'be the same as the size of the truss');
        error(str);
      end
      if ~all(numvectors | symvectors | using_default, 'all')
        str = stack('the contents of the cell array', ...
                    'for the reference points must', ...
                    'contain numeric or symbolic vectors', ...
                    'or the string ''default''');
        error(str);
      end
\end{verbatim}


\subsection*{case for non-scalar trusses}

\begin{verbatim}
      if isempty(t)
        [us ms ls] = deal(struct.empty);
        [ua ma la] = deal(sym.empty);
        return;
      elseif ~isscalar(t)
        try
          [~, Mode Reference] = scalar_expand(t, Mode, Reference);
          [us ua ms ma ls la] = deal(cell(size(t)));
          for k = 1:numel(t)
            [us{k} ua{k} ms{k} ma{k} ls{k} la{k}] = ...
            t(k).solve('Mode', Mode{k}, 'Reference', Reference{k});
          end
          return;
        catch Error
          str = 'for the truss with linear index %d:\n%s';
          new_Error = MException('', str, k, Error.message);
          if ~isempty(Error.cause)
            throw(addCause(new_Error, Error.cause{1}));
          else
            throw(new_Error);
          end
        end
      else
        Reference = Reference{1};
      end
\end{verbatim}


\subsection*{convert the table variables to cell arrays of symbolic arrays}

\begin{verbatim}
      t = t.convert2cellsym('Mode', 'nan');
      t.load.magnitude = cellfun(@formula, t.load.magnitude, uniform{:});
      t.load.distance = cellfun(@formula, t.load.distance, uniform{:});
      t.joint.distance = cellfun(@formula, t.joint.distance, uniform{:});
      t.member.start_distance = cellfun(@formula, ...
                                        t.member.start_distance, ...
                                        uniform{:});
      t.member.end_distance = cellfun(@formula, ...
                                      t.member.end_distance, ...
                                      uniform{:});
      t.member.E = cellfun(@formula, t.member.E, uniform{:});
      t.member.A = cellfun(@formula, t.member.A, uniform{:});
\end{verbatim}


\subsection*{compute these useful values}

\begin{par}
frequrently used arrays
\end{par} \vspace{1em}
\begin{verbatim}
      IAC = {'IgnoreAnalyticConstraints' true};
      Full = {'FactorMode' 'full'};
      Complex = {'FactorMode' 'complex'};
      Real = {'FactorMode' 'real'};
      Expand = {'Expand' true};
      iA_str = 'unable to solve the truss';
      % load table variables
      t.load = sortrows(t.load, 'class');
      class = t.load.class;
      load_fieldnames = compute_fieldnames('load', t);
      load_magnitude = t.load.magnitude;
      load_distance = t.load.distance;
      num_loads = height(t.load);
      % joint table variable
      joint_fieldames = compute_fieldnames('joint', t);
      joint_distance = t.joint.distance;
      num_joints = height(t.joint);
      % member table variables
      member_fieldnames = compute_fieldnames('member', t);
      start_distance = t.member.start_distance;
      end_distance = t.member.end_distance;
      E = t.member.E;
      A = t.member.A;
      num_members = height(t.member);
      % class locations
      reaction_loc = class == "reaction";
      concentrated_loc = class == "concentrated";
      % reaction loads table
      reaction = t.load(reaction_loc,:);
      reaction_vars = symvar(sym(convert2col(reaction.magnitude)));
      num_reactions = height(reaction);
      num_reaction_vars = length(reaction_vars);
      % concentrated loads table
      concentrated = t.load(concentrated_loc,:);
      concentrated_vars = symvar(sym(convert2col(concentrated.magnitude)));
      % column vector of magnitudes, distances, and lengths
      column.m = sym(convert2col(load_magnitude));
      column.d = sym([convert2col(load_distance);
                      convert2col(joint_distance);
                      convert2col(start_distance);
                      convert2col(end_distance)]);
      column.lengths = [cellfun(@length, load_magnitude);
                        cellfun(@length, load_distance);
                        cellfun(@length, joint_distance);
                        cellfun(@length, start_distance);
                        cellfun(@length, end_distance)];
      % truss dimensions and variables to exclude
      if isempty(column.lengths)
        num_dimensions = 1;
      else
        num_dimensions = column.lengths(1);
      end
      Vars2Exclude = symvar([column.m; column.d;
                             convert2col(E); convert2col(A)]);
      % scalar cell array locations
      func = @(arg) isScalar(arg) && isallfinite(arg);
      scalar_Es = cellfun(func, E);
      scalar_As = cellfun(func, A);
      % vector cell array locations
      func = @(arg) isVector(arg, 'CheckEmpty', true) && ...
                    isallfinite(arg);
      vector_load_magnitudes = cellfun(func, load_magnitude);
      vector_load_distances = cellfun(func, load_distance);
      vector_joint_distances = cellfun(func, joint_distance);
      vector_start_distances = cellfun(func, start_distance);
      vector_end_distances = cellfun(func, end_distance);
      vector_distances = all(vector_load_distances) && ...
                         all(vector_joint_distances) && ...
                         all(vector_start_distances) && ...
                         all(vector_end_distances);
      % symbolic variable cell array locations
      func = @(arg) issymvarmultiplevector(arg, 'CountZero', true);
      symvarmultiplevectors = cellfun(func, load_magnitude);
\end{verbatim}


\subsection*{inspect the magnitudes and distances}

\begin{par}
check the magnitudes of the reactions
\end{par} \vspace{1em}
\begin{verbatim}
      if any(reaction_loc & ~symvarmultiplevectors)
        str = stack('the magnitudes of the reactions', ...
                    'must be numeric or symbolic vectors', ...
                    'containing numeric scalar multiples', ...
                    'of a symbolic variable scalar');
        error(str);
      end
      % check the magnitudes of the concentrated loads
      if any(concentrated_loc & ~vector_load_magnitudes)
        str = stack('the magnitudes of the concentrated loads', ...
                    'must be non-empty numeric or symbolic scalars', ...
                    'with no Inf or NaN values');
        error(str);
      end
      % check the distances
      if ~vector_distances
        str = stack('the distances must be', ...
                    'non-empty numeric or symbolic vectors', ...
                    'with no Inf or NaN values');
        error(str);
      end
      % check the lengths of the magnitudes and distances
      if ~isallequal(column.lengths) || any(column.lengths > 3)
        str = stack('the lengths of the magnitudes and distances', ...
                    'must all be the same and not exceed 3');
        error(str);
      end
      % check E and A
      if ~all(scalar_Es & scalar_As)
        str = stack('''E'' and ''A'' must be', ...
                    'numeric or symbolic scalars', ...
                    'with no Inf or NaN values');
        error(str);
      end
      % the distances must have compatible units
      if ~checkUnits(sum(column.d), 'Compatible')
        error('the distances must have compatible units');
      end
      % the distances must not contain
      % the magnitude of the reactions
      if any(ismember(symvar(column.d), reaction_vars))
        str = stack('the distances must not contain', ...
                    'the magnitudes of the reactions');
        error(str);
      end
\end{verbatim}


\subsection*{inspect the loads}

\begin{par}
the magnitudes of the reactions must not be scalar multiples of one another
\end{par} \vspace{1em}
\begin{verbatim}
      reaction.magnitude = sym(convert2row(reaction.magnitude));
      magnitude_col = reaction.magnitude(:);
      magnitude_col(~issymvarmultiple(magnitude_col)) = [];
      if ~isequallen(magnitude_col, reaction_vars)
        str = stack('the magnitudes of the reactions', ...
                    'must not be scalar multiples of one another');
        error(str);
      end
      % the magnitudes of the concentrated loads
      % must not contain the magnitudes of the reactions
      if any(ismember(concentrated_vars, reaction_vars))
        str = stack('the magnitudes of the concentrated loads', ...
                    'must not contain the magnitudes of the reactions');
        error(str);
      end
      % there must be unique distances values for the reactions
      reaction.distance = sym(convert2row(reaction.distance));
      for k = 1:num_reactions-1
        lhs = reaction.distance(k,:);
        for p = k+1:num_reactions
          rhs = reaction.distance(p,:);
          if isAlwaysError(lhs == rhs, iA_str)
            error('at least 2 reactions have the same distance');
          end
        end
      end
      % the truss must be stable
      Args = {num_reaction_vars num_dimensions num_joints num_members};
      if ~Truss.stable(Args{:})
        error('the truss must be stable');
      end
\end{verbatim}


\subsection*{inspect the joints}

\begin{par}
the length of the reference point must be the same as the number of truss dimensions
\end{par} \vspace{1em}
\begin{verbatim}
      if using_default
        Reference = joint_distance{1};
      end
      if ~isrow(Reference) && ~isscalar(Reference)
        Reference = Reference.';
      end
      if ~islen(Reference, num_dimensions)
        str = stack('the length of the reference point', ...
                    'must be the same as', ...
                    'the number of truss dimensions');
        error(str);
      end
      % the reference point must have units
      % that are compatible with the distance units
      if ~checkUnits(sum(Reference)+sum(column.d), 'Compatible')
        str = stack('the reference point must have units', ...
                    'that are compatible with the distance units');
        error(str);
      end
      % there must be unique distance values for the joints
      joint_distance = sym(convert2row(joint_distance));
      for k = 1:num_joints-1
        for p = k+1:num_joints
          equal_distance = joint_distance(k,:) == joint_distance(p,:);
          if isAlwaysError(equal_distance, iA_str)
            fields = [t.joint.Properties.RowNames(k);
                      t.joint.Properties.RowNames(p)];
            str = stack('joints ''%s'' and ''%s''', ...
                        'have the same distance');
            error(str, fields{:});
          end
        end
      end
      % all loads must be connected to a joint
      load_magnitude = sym(convert2row(load_magnitude));
      load_distance = sym(convert2row(load_distance));
      for k = 1:num_loads
        load_at_joint = load_distance(k,:) == joint_distance;
        load_at_joint = all(isAlwaysError(load_at_joint, iA_str), 2);
        if ~any(load_at_joint)
          error('load #%d is not connected to a joint', k);
        end
      end
\end{verbatim}


\subsection*{inspect the members}

\begin{par}
there must be unique distance values for the members
\end{par} \vspace{1em}
\begin{verbatim}
      for k = 1:num_members-1
        for p = k+1:num_members
          equal_distance = (start_distance{k} == start_distance{p}) & ...
                           (end_distance{k} == end_distance{p});
          if isAlwaysError(equal_distance, iA_str)
            str = stack('members ''%s'' and ''%s''', ...
                        'have the same distances');
            error(str, member_fieldnames{[k p]});
          end
        end
      end
      % both sides of all members
      % must have different distances and
      % must be connected to a joint
      start_distance = sym(convert2row(start_distance));
      end_distance = sym(convert2row(end_distance));
      for k = 1:num_members
        start_equals_end = start_distance(k,:) == end_distance(k,:);
        start_equals_end = isAlwaysError(start_equals_end, iA_str);
        start_equals_end = all(start_equals_end, 2);
        start_at_joint = start_distance(k,:) == joint_distance;
        start_at_joint = all(isAlwaysError(start_at_joint, iA_str), 2);
        end_at_joint = end_distance(k,:) == joint_distance;
        end_at_joint = all(isAlwaysError(end_at_joint, iA_str), 2);
        if any(start_equals_end)
          str = stack('both sides of member ''%s''', ...
                      'have the same distance');
        elseif ~any(start_at_joint) && ~any(end_at_joint)
          str = stack('both sides of member ''%s''', ...
                      'are not connected to a joint');
        elseif ~any(start_at_joint)
          str = stack('the starting side of member ''%s''', ...
                      'is not connected to a joint');
        elseif ~any(end_at_joint)
          str = stack('the ending side of member ''%s''', ...
                      'is not connected to a joint');
        else
          str = '';
        end
        if ~isempty(str)
          error(str, member_fieldnames{k});
        end
      end
\end{verbatim}


\subsection*{compute the global stiffness matrix data}

\begin{verbatim}
      Ind = @(k) num_dimensions*(k-1)+1;
      N = 0:num_dimensions-1;
      distance = end_distance-start_distance;
      lambda = simplify(unit_vector(distance, 'Dim', 2), IAC{:});
      E = sym(E);
      A = sym(A);
      L = simplify(Norm(distance, 'Dim', 2), IAC{:});
      EAL = simplify(E.*A./L, IAC{:});
\end{verbatim}


\subsection*{compute the global stiffness matrix}

\begin{verbatim}
      connect = zeros(num_members, 2*num_dimensions);
      [T Kl Km] = deal(cell(num_members, 1));
      Kg = sym.zeros(num_dimensions*num_joints);
      for k = 1:num_members
        % joint indices
        start_at_joint = start_distance(k,:) == joint_distance;
        start_at_joint = all(isAlwaysError(start_at_joint, iA_str), 2);
        end_at_joint = end_distance(k,:) == joint_distance;
        end_at_joint = all(isAlwaysError(end_at_joint, iA_str), 2);
        joint_ind = [find(start_at_joint, 1) find(end_at_joint, 1)];
        % connectivity matrix
        connect(k,1:num_dimensions) = Ind(joint_ind(1))+N;
        connect(k,num_dimensions+1:2*num_dimensions) = Ind(joint_ind(2))+N;
        % member stiffness matrix
        lambda_vals = lambda(k,:);
        zero_vals = zeros(size(lambda_vals));
        T{k} = [lambda_vals zero_vals; zero_vals lambda_vals];
        Kl{k} = EAL(k)*[1 -1; -1 1];
        Km{k} = T{k}.'*Kl{k}*T{k};
        % global stiffness matrix
        ind = connect(k,:);
        Kg(ind,ind) = Kg(ind,ind)+Km{k};
      end
\end{verbatim}


\subsection*{compute the global displacement and force vectors}

\begin{verbatim}
      dim = [num_dimensions*num_joints 1];
      Ug = randsym(dim, 'Vars2Exclude', Vars2Exclude);
      Fg = sym.zeros(num_dimensions*num_joints, 1);
      var_found = issymvarmultiple(load_magnitude);
      for k = 1:num_joints
        equal_distance = load_distance == joint_distance(k,:);
        equal_distance = isAlwaysError(all(equal_distance, 2), iA_str);
        ind = Ind(k);
        for p = 1:num_dimensions
          % global displacement vector
          reaction_found = reaction_loc & equal_distance & var_found(:,p);
          if any(reaction_found)
            Ug(ind) = 0;
          end
          % global force vector
          Fg(ind) = sum(load_magnitude(equal_distance, p));
          ind = ind+1;
        end
      end
\end{verbatim}


\subsection*{solve for the unknown variables}

\begin{par}
compute the solution
\end{par} \vspace{1em}
\begin{verbatim}
      eqn = Kg*Ug == Fg;
      unknowns = [reaction_vars Ug(issymvar(Ug)).'];
      soln = solve(eqn, unknowns);
      Ug = subs(Ug, soln);
      % compute the joint displacement structure array
      fields = joint_fieldames;
      for k = 1:length(fields)
        if isempty(Ug)
          us.(fields{k}) = Ug;
          continue;
        end
        us.(fields{k}) = simplify(Ug(k), IAC{:});
        switch Mode
          case "factor"
            us.(fields{k}) = prodfactor(us.(fields{k}));
          case "factor full"
            us.(fields{k}) = prodfactor(us.(fields{k}), Full{:});
          case "factor complex"
            us.(fields{k}) = prodfactor(us.(fields{k}), Complex{:});
          case "factor real"
            us.(fields{k}) = prodfactor(us.(fields{k}), Real{:});
          case "simplify fraction"
            us.(fields{k}) = simplifyFraction(us.(fields{k}));
          case "simplify fraction expand"
            us.(fields{k}) = simplifyFraction(us.(fields{k}), Expand{:});
        end
      end
      % compute the force and moment structure array
      num_fields = num_members+num_loads;
      zero_vals = zeros(num_fields, 3-num_dimensions);
      Reference = repmat(Reference, num_fields, 1);
      magnitude = subs(load_magnitude, soln);
      distance = [[start_distance; load_distance]-Reference zero_vals];
      fields = [member_fieldnames; load_fieldnames];
      for k = 1:length(fields)
        if isempty(Ug)
          f.m.(fields{k}) = Ug;
          f.c.(fields{k}) = Ug;
          m.m.(fields{k}) = Ug;
          m.c.(fields{k}) = Ug;
          continue;
        end
        if k <= num_members
          % member force magnitude
          ind = connect(k,:);
          f.m.(fields{k}) = Kl{k}*T{k}*Ug(ind);
          f.m.(fields{k}) = simplify(f.m.(fields{k})(2), IAC{:});
          % member force components
          f.c.(fields{k}) = Km{k}*Ug(ind);
          f.c.(fields{k}) = [f.c.(fields{k})(Ind(2):end); zero_vals(k,:)];
          f.c.(fields{k}) = simplify(f.c.(fields{k}), IAC{:});
        else
          % load force magnitude
          f.m.(fields{k}) = Norm(magnitude(k-num_members,:));
          f.m.(fields{k}) = simplify(f.m.(fields{k}), IAC{:});
          % load force components
          f.c.(fields{k}) = [magnitude(k-num_members,:) zero_vals(k,:)].';
          f.c.(fields{k}) = simplify(f.c.(fields{k}), IAC{:});
        end
        % moment components
        m.c.(fields{k}) = cross(distance(k,:).', f.c.(fields{k}));
        m.c.(fields{k}) = simplify(m.c.(fields{k}), IAC{:});
        % moment magnitudes
        m.m.(fields{k}) = Norm(m.c.(fields{k}));
        m.m.(fields{k}) = simplify(m.m.(fields{k}), IAC{:});
        switch Mode
          case "factor"
            f.m.(fields{k}) = prodfactor(f.m.(fields{k}));
            f.c.(fields{k}) = prodfactor(f.c.(fields{k}));
            m.m.(fields{k}) = prodfactor(m.m.(fields{k}));
            m.c.(fields{k}) = prodfactor(m.c.(fields{k}));
          case "factor full"
            f.m.(fields{k}) = prodfactor(f.m.(fields{k}), Full{:});
            f.c.(fields{k}) = prodfactor(f.c.(fields{k}), Full{:});
            m.m.(fields{k}) = prodfactor(m.m.(fields{k}), Full{:});
            m.c.(fields{k}) = prodfactor(m.c.(fields{k}), Full{:});
          case "factor complex"
            f.m.(fields{k}) = prodfactor(f.m.(fields{k}), Complex{:});
            f.c.(fields{k}) = prodfactor(f.c.(fields{k}), Complex{:});
            m.m.(fields{k}) = prodfactor(m.m.(fields{k}), Complex{:});
            m.c.(fields{k}) = prodfactor(m.c.(fields{k}), Complex{:});
          case "factor real"
            f.m.(fields{k}) = prodfactor(f.m.(fields{k}), Real{:});
            f.c.(fields{k}) = prodfactor(f.c.(fields{k}), Real{:});
            m.m.(fields{k}) = prodfactor(m.m.(fields{k}), Real{:});
            m.c.(fields{k}) = prodfactor(m.c.(fields{k}), Real{:});
          case "simplify fraction"
            f.m.(fields{k}) = simplifyFraction(f.m.(fields{k}));
            f.c.(fields{k}) = simplifyFraction(f.c.(fields{k}));
            m.m.(fields{k}) = simplifyFraction(m.m.(fields{k}));
            m.c.(fields{k}) = simplifyFraction(m.c.(fields{k}));
          case "simplify fraction expand"
            f.m.(fields{k}) = simplifyFraction(f.m.(fields{k}), Expand{:});
            f.c.(fields{k}) = simplifyFraction(f.c.(fields{k}), Expand{:});
            m.m.(fields{k}) = simplifyFraction(m.m.(fields{k}), Expand{:});
            m.c.(fields{k}) = simplifyFraction(m.c.(fields{k}), Expand{:});
        end
      end
      % compute the member force and moment structure array
      ms.f.m = rmfield(f.m, setdiff(fields, member_fieldnames));
      ms.f.c = rmfield(f.c, setdiff(fields, member_fieldnames));
      ms.m.m = rmfield(m.m, setdiff(fields, member_fieldnames));
      ms.m.c = rmfield(m.c, setdiff(fields, member_fieldnames));
      % compute the reaction force and moment structure array
      ls.f.m = rmfield(f.m, setdiff(fields, load_fieldnames));
      ls.f.c = rmfield(f.c, setdiff(fields, load_fieldnames));
      ls.m.m = rmfield(m.m, setdiff(fields, load_fieldnames));
      ls.m.c = rmfield(m.c, setdiff(fields, load_fieldnames));
      % compute the symbolic arrays
      s2c = @struct2cell;
      if ~isempty(Ug)
        % joint displacement symbolic array
        ua = [sym(fieldnames(us)) sym(s2c(us))];
        % member force and moment symbolic arrays
        ma.f.m = [sym(fieldnames(ms.f.m)) sym(s2c(ms.f.m))];
        ma.f.c = [sym(fieldnames(ms.f.c)) sym(convert2row(s2c(ms.f.c)))];
        ma.m.m = [sym(fieldnames(ms.m.m)) sym(s2c(ms.m.m))];
        ma.m.c = [sym(fieldnames(ms.m.c)) sym(convert2row(s2c(ms.m.c)))];
        % load force and moment symbolic arrays
        la.f.m = [sym(fieldnames(ls.f.m)) sym(s2c(ls.f.m))];
        la.f.c = [sym(fieldnames(ls.f.c)) sym(convert2row(s2c(ls.f.c)))];
        la.m.m = [sym(fieldnames(ls.m.m)) sym(s2c(ls.m.m))];
        la.m.c = [sym(fieldnames(ls.m.c)) sym(convert2row(s2c(ls.m.c)))];
      else
        % joint displacement symbolic array
        ua = [sym(fieldnames(us)) nan(numFields(us), 1)];
        % member force and moment symbolic arrays
        ma.f.m = [sym(fieldnames(ms.f.m)) nan(numFields(ms.f.m), 1)];
        ma.f.c = [sym(fieldnames(ms.f.c)) nan(numFields(ms.f.c), 1)];
        ma.m.m = [sym(fieldnames(ms.m.m)) nan(numFields(ms.m.m), 1)];
        ma.m.c = [sym(fieldnames(ms.m.c)) nan(numFields(ms.m.c), 1)];
        % load force and moment symbolic arrays
        la.f.m = [sym(fieldnames(ls.f.m)) nan(numFields(ls.f.m), 1)];
        la.f.c = [sym(fieldnames(ls.f.c)) nan(numFields(ls.f.c), 1)];
        la.m.m = [sym(fieldnames(ls.m.m)) nan(numFields(ls.m.m), 1)];
        la.m.c = [sym(fieldnames(ls.m.c)) nan(numFields(ls.m.c), 1)];
      end
\end{verbatim}
\begin{verbatim}
    end
    % ==
  end
  % ==
  methods (Static) % truss calculation static methods
    % ==
    function [bool diff] = stable(R, D, J, M)
\end{verbatim}
\begin{verbatim}
      % ----------------------------------------
      % - returns true if the truss is stable
      % - the truss is stable
      %   if R >= D*J-M
      % - where:
      %   R == the number of reaction magnitudes
      %   D == the number of truss dimensions
      %   J == the number of joints
      %   M == the number of members
      %   R >= 0, 1 <= D <= 3, J >= 0, M >= 0
      % ----------------------------------------
\end{verbatim}


\subsection*{check the input arguments}

\begin{par}
check the argument classes
\end{par} \vspace{1em}
\begin{verbatim}
      arguments
        R double {mustBeNonempty, mustBeInteger, mustBeNonnegative};
        D double {mustBeNonempty, mustBeInteger, mustBeInRange(D, 1, 3)};
        J double {mustBeNonempty, mustBeInteger, mustBeNonnegative};
        M double {mustBeNonempty, mustBeInteger, mustBeNonnegative};
      end
      % check the argument dimensions
      if ~compatible_dims(R, D, J, M)
        error('input arguments must have compatible dimensions');
      end
\end{verbatim}


\subsection*{determine if the truss is stable}

\begin{verbatim}
      lhs = R;
      rhs = D.*J-M;
      uniform = {'UniformOutput' false};
      func = @(arg) arg == 0;
      zero_args = fold(@or, cellfun(func, {R J M}, uniform{:}));
      invalid_args = J < 2;
      bool = lhs >= rhs;
      bool(zero_args | invalid_args) = false;
      diff = lhs-rhs;
\end{verbatim}
\begin{verbatim}
    end
    % ==
    function bool = statically_determinate(R, D, J, M)
\end{verbatim}
\begin{verbatim}
      % ----------------------------------------
      % - returns true if the truss
      %   is statically determinate
      % - the truss is
      %   statically determinate
      %   if R == D*J-M
      % - where:
      %   R == the number of reaction magnitudes
      %   D == the number of truss dimensions
      %   J == the number of joints
      %   M == the number of members
      %   R >= 0, 1 <= D <= 3, J >= 0, M >= 0
      % ----------------------------------------
\end{verbatim}


\subsection*{check the input arguments}

\begin{par}
check the argument classes
\end{par} \vspace{1em}
\begin{verbatim}
      arguments
        R double {mustBeNonempty, mustBeInteger, mustBeNonnegative};
        D double {mustBeNonempty, mustBeInteger, mustBeInRange(D, 1, 3)};
        J double {mustBeNonempty, mustBeInteger, mustBeNonnegative};
        M double {mustBeNonempty, mustBeInteger, mustBeNonnegative};
      end
      % check the argument dimensions
      if ~compatible_dims(R, D, J, M)
        error('input arguments must have compatible dimensions');
      end
\end{verbatim}


\subsection*{determine if the truss is statically determinate}

\begin{verbatim}
      lhs = R;
      rhs = D.*J-M;
      uniform = {'UniformOutput' false};
      func = @(arg) arg == 0;
      zero_args = fold(@or, cellfun(func, {R J M}, uniform{:}));
      invalid_args = J < 2;
      bool = lhs == rhs;
      bool(zero_args | invalid_args) = false;
\end{verbatim}
\begin{verbatim}
    end
    % ==
  end
  % ==
end
% ==
function fields = compute_fieldnames(Prop, t)
  % -------------------------------
  % - helper function for computing
  %   the field names of the struct
  %   for the truss tables
  % -------------------------------
  uniform = {'UniformOutput' false};
  switch Prop
    case "load"
      reaction_loc = t.load.class == 'reaction';
      concentrated_loc = t.load.class == 'concentrated';
      num_reactions = height(t.load(reaction_loc,:));
      num_concentrated = height(t.load(concentrated_loc,:));
      fields = t.load.Properties.RowNames;
      if isempty(fields)
        Nums = [{num2cellstr(1:num_reactions)} ...
                {num2cellstr(1:num_concentrated)}];
        Cell = [{repmat({'R'}, 1, num_reactions)} ...
                {repmat({'P'}, 1, num_concentrated)}];
        for k = 1:2
          Cell{k} = cellfun(@horzcat, Cell{k}, Nums{k}, uniform{:});
        end
        fields = vertcat(Cell{:});
      end
    case "joint"
      num_joints = height(t.joint);
      if isempty(t.joint.distance)
        num_dimensions = 1;
      else
        num_dimensions = length(t.joint.distance{1});
      end
      fields = t.joint.Properties.RowNames;
      switch num_dimensions
        case 1
          Cell = repmat({'u'}, num_joints, 1);
        case 2
          Cell = repmat({'u'; 'v'}, num_joints, 1);
        case 3
          Cell = repmat({'u'; 'v'; 'w'}, num_joints, 1);
      end
      if isempty(fields)
        fields = num2cellstr(repmat(1:num_joints, num_dimensions, 1));
      else
        fields = repmat(fields(:).', num_dimensions, 1);
      end
      fields = cellfun(@horzcat, Cell, fields(:), uniform{:});
    case "member"
      num_members = height(t.member);
      fields = t.member.Properties.RowNames;
      if isempty(fields)
        Nums = num2cellstr(1:num_members).';
        fields = repmat({'member'}, num_members, 1);
        fields = cellfun(@horzcat, fields, Nums, uniform{:});
      end
      for k = 1:num_members
        while ismember(fields{k}, t.load.Properties.RowNames)
          fields{k} = [fields{k} '_'];
        end
      end
  end
  fields = convert2identifier(fields);
end
% ==
\end{verbatim}

        \color{lightgray} \begin{verbatim}
ans = 

  Truss with properties:

                load: [0×3 table]
               joint: [0×1 table]
              member: [0×4 table]
       load_varnames: {3×1 cell}
      joint_varnames: {'distance'}
     member_varnames: {4×1 cell}
    class_categories: {2×1 cell}

\end{verbatim} \color{black}
    


\end{document}

