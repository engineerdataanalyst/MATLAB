
% This LaTeX was auto-generated from MATLAB code.
% To make changes, update the MATLAB code and republish this document.

\documentclass{article}
\usepackage{graphicx}
\usepackage{color}

\sloppy
\definecolor{lightgray}{gray}{0.5}
\setlength{\parindent}{0pt}

\begin{document}

    
    
\subsection*{Contents}

\begin{itemize}
\setlength{\itemsep}{-1ex}
   \item construct the text
   \item construct the words table
   \item construct the symbols table
   \item construct the letters table
   \item parse the document if requested
   \item compute the heading
   \item print the document
   \item check the input arguments
   \item case for non-scalar documents
   \item reset the parsing tables
   \item check the input arguments
   \item case for non-scalar documents
   \item parse the document
   \item check the input arguments
   \item case for non-scalar documents
   \item compute the word information from the table
   \item find the words in the text matching the target word
   \item modify the outputs to a more convenient type
   \item check the input arguments
   \item find the bad words
\end{itemize}
\begin{verbatim}
classdef document < matlab.mixin.Copyable & matlab.mixin.CustomDisplay
  % ==
  % --------------------
  % - the document class
  % --------------------
  % ==
  properties % text property
    text {mustBeTextScalar} = '';
  end
  % ==
  properties (Dependent, SetAccess = private) % number of lines property
    num_lines;
  end
  % ==
  properties (SetAccess = private) % parsing tables
    words;
    symbols;
    letters;
  end
  % ==
  properties (Constant) % bad words property
    bad_words = {'fuck'; 'shit'; 'bitch'; 'dick'; 'ass';
                 'pussy'; 'damn'; 'nigger'; 'nigga'};
  end
  % ==
  methods % document constructor
    % ==
    function doc = document(text, parse_doc)
\end{verbatim}
\begin{verbatim}
      % --------------------------
      % - the document constructor
      % --------------------------
\end{verbatim}


\subsection*{construct the text}

\begin{verbatim}
      arguments
        text {mustBeTextScalar} = '';
        parse_doc (1,1) logical = false;
      end
      if nargin == 1
        parse_doc = true;
      end
      doc.text = text;
\end{verbatim}


\subsection*{construct the words table}

\begin{verbatim}
      persistent tbl1;
      if ~istable(tbl1)
        tbl1.count = 0;
        tbl1 = struct2table(tbl1, 'RowNames', {'0'});
        tbl1 = addprop(tbl1, {'s' 'f'}, {'table' 'table'});
        tbl1 = addprop(tbl1, 'num_words', 'variable');
        tbl1.Properties.DimensionNames{1} = 'word';
        tbl1.Properties.CustomProperties.s = {};
        tbl1.Properties.CustomProperties.f = {};
        tbl1.Properties.CustomProperties.num_words = 0;
        tbl1(:,:) = [];
      end
      doc.words = tbl1;
\end{verbatim}


\subsection*{construct the symbols table}

\begin{verbatim}
      persistent tbl2;
      if ~istable(tbl2)
        tbl2.count = 0;
        tbl2 = struct2table(tbl2, 'RowNames', {'0'});
        tbl2 = addprop(tbl2, 'num_symbols', 'variable');
        tbl2.Properties.DimensionNames{1} = 'symbol';
        tbl2.Properties.CustomProperties.num_symbols = 0;
        tbl2(:,:) = [];
      end
      doc.symbols = tbl2;
\end{verbatim}


\subsection*{construct the letters table}

\begin{verbatim}
      persistent tbl3;
      if ~istable(tbl3)
        tbl3.upper = zeros(26,1);
        tbl3.lower = zeros(26,1);
        tbl3 = struct2table(tbl3, 'RowNames', num2cell('A':'Z'));
        tbl3 = addprop(tbl3, 'num_letters', 'variable');
        tbl3.Properties.DimensionNames{1} = 'letter';
        tbl3.Properties.CustomProperties.num_letters = [0 0];
      end
      doc.letters = tbl3;
\end{verbatim}


\subsection*{parse the document if requested}

\begin{verbatim}
      if parse_doc
        doc.parse;
      end
\end{verbatim}
\begin{verbatim}
    end
    % ==
    function set.text(doc, text)
      % --------------------------
      % - removes any '\r'
      %   characters from the text
      % --------------------------
      if isstring(text)
        text{1}(text{1} == sprintf('\r')) = [];
      else
        text(text == sprintf('\r')) = [];
      end
      doc.text = text;
    end
    % ==
    function num_lines = get.num_lines(doc)
      % ------------------------
      % - computes the number of
      %   lines in the text
      % ------------------------
      if isempty(doc.text)
        num_lines = 0;
      else
        num_lines = count(doc.text, newline)+1;
      end
    end
    % ==
  end
  % ==
  methods (Sealed, Access = protected) % displaying methods
    % ==
    function displayEmptyObject(doc)
      % --------------------------------
      % - override of the display method
      %   for nicely displaying
      %   an empty document
      % --------------------------------
      empty_str = ['empty ' class(doc) ' array'];
      line = repmat('-', size(empty_str));
      fprintf('%s\n%s\n%s\n\n', line, empty_str, line);
    end
    % ==
    function displayScalarObject(doc)
\end{verbatim}
\begin{verbatim}
      % --------------------------------
      % - override of the display method
      %   for nicely displaying
      %   a scalar document
      % --------------------------------
\end{verbatim}


\subsection*{compute the heading}

\begin{verbatim}
      num_str = num2str(doc.num_lines);
      heading = [class(doc) ' with ' num_str ' lines'];
      line = repmat('-', size(heading));
      if doc.num_lines == 1
        heading(end) = [];
        line(end) = [];
      end
      heading = stack(line, heading, line);
\end{verbatim}


\subsection*{print the document}

\begin{verbatim}
      if isempty(doc.text)
        fprintf('%s\n\n', heading);
      else
        fprintf('%s\n%s\n\n', heading, doc.text);
      end
\end{verbatim}

        \color{lightgray} \begin{verbatim}---------------------
document with 0 lines
---------------------

\end{verbatim} \color{black}
    \begin{verbatim}
    end
    % ==
    function displayNonScalarObject(doc)
      % --------------------------------
      % - override of the display method
      %   for nicely displaying
      %   a non-scalar document
      % --------------------------------
      size_cell = num2cell(size(doc));
      doc_str = repmat('%dx', size(size_cell));
      doc_str(end) = [];
      doc_str = [sprintf(doc_str, size_cell{:}) ' ' class(doc) ' array'];
      line = repmat('-', size(doc_str));
      fprintf('%s\n%s\n%s\n\n', line, doc_str, line);
    end
    % ==
  end
  % ==
  methods % parsing methods
    % ==
    function reset(doc, options)
\end{verbatim}
\begin{verbatim}
      % ---------------------------
      % - resets the parsing tables
      %   of the document class
      % ---------------------------
\end{verbatim}


\subsection*{check the input arguments}

\begin{par}
check the argument classes
\end{par} \vspace{1em}
\begin{verbatim}
      arguments
        doc;
      end
      arguments
        options.Prop (1,:) ...
        {mustBeText, ...
         mustBeMemberi(options.Prop, ["words" "symbols" "letters"])};
      end
      % check the property option
      if isfield(options, 'Prop')
        Prop = unique(string(lower(options.Prop)), 'stable');
      else
        Prop = ["words" "symbols" "letters"];
      end
\end{verbatim}


\subsection*{case for non-scalar documents}

\begin{verbatim}
      if isempty(doc)
        return;
      elseif ~isscalar(doc)
        for k = 1:numel(doc)
          doc(k).reset('Prop', Prop);
        end
        return;
      end
\end{verbatim}


\subsection*{reset the parsing tables}

\begin{verbatim}
      for k = Prop
        switch k
          case "words"
            % reset the words table
            if ~isempty(doc.words)
              doc.words(:,:) = [];
              doc.words.Properties.CustomProperties.s = {};
              doc.words.Properties.CustomProperties.f = {};
              doc.words.Properties.CustomProperties.num_words = 0;
            end
          case "symbols"
            % reset the symbols table
            if ~isempty(doc.symbols)
              doc.symbols(:,:) = [];
              doc.symbols.Properties.CustomProperties.num_symbols = 0;
            end
          case "letters"
            % reset the letters table
            if any(doc.letters{:,:}, 'all')
              doc.letters.upper(doc.letters.upper ~= 0) = 0;
              doc.letters.lower(doc.letters.lower ~= 0) = 0;
              doc.letters.Properties.CustomProperties.num_letters = [0 0];
            end
        end
      end
\end{verbatim}
\begin{verbatim}
    end
    % ==
    function parse(doc, options)
\end{verbatim}
\begin{verbatim}
      % ------------------------------
      % - parses the document for
      %   the following information:
      % - 1.) number of words
      %   2.) number of symbols
      %   3.) number of letters
      %      (uppercase and lowercase)
      % ------------------------------
\end{verbatim}


\subsection*{check the input arguments}

\begin{par}
check the argument classes
\end{par} \vspace{1em}
\begin{verbatim}
      arguments
        doc;
      end
      arguments
        options.Prop (1,:) ...
        {mustBeText, ...
         mustBeMemberi(options.Prop, ["words" "symbols" "letters"])};
      end
      % check the property option
      if isfield(options, 'Prop')
        Prop = unique(string(lower(options.Prop)), 'stable');
      else
        Prop = ["words" "symbols" "letters"];
      end
\end{verbatim}


\subsection*{case for non-scalar documents}

\begin{verbatim}
      if isempty(doc)
        return;
      elseif ~isscalar(doc)
        for k = 1:numel(doc)
          doc(k).parse('Prop', Prop);
        end
        return;
      end
\end{verbatim}


\subsection*{parse the document}

\begin{verbatim}
      doc.reset('Prop', Prop);
      for k = Prop
        switch k
          case "words"
            % parse the words
            [words_in_text s f] = regexp(doc.text, '(\w|[-''])*', 'match');
            unique_words = unique(words_in_text, 'stable').';
            if ~isempty(unique_words)
              word.count = zeros(size(unique_words));
              word = struct2table(word, 'RowNames', unique_words);
              doc.words = [doc.words; word];
              Cell = cell(size(unique_words));
              doc.words.Properties.CustomProperties.s = Cell;
              doc.words.Properties.CustomProperties.f = Cell;
            end
            for p = 1:length(unique_words)
              ind = strcmp(words_in_text, unique_words{p});
              doc.words.count(p) = nnz(ind);
              doc.words.Properties.CustomProperties.s{p} = s(ind).';
              doc.words.Properties.CustomProperties.f{p} = f(ind).';
            end
            num = sum(doc.words.count);
            doc.words.Properties.CustomProperties.num_words = num;
          case "symbols"
            % parse the symbols
            if isstring(doc.text)
              new_text = char(doc.text);
            else
              new_text = doc.text;
            end
            ind = ~isupper(new_text) & ~islower(new_text) & ...
                  ~isspace(new_text);
            symbols_in_text = new_text(ind);
            unique_symbols = num2cell(unique(symbols_in_text, 'stable')).';
            colons = cellfun(@(arg)strcmp(arg, ':'), unique_symbols);
            unique_symbols(colons) = {':_'};
            if ~isempty(unique_symbols)
              symbol.count = zeros(size(unique_symbols));
              symbol = struct2table(symbol, 'RowNames', unique_symbols);
              doc.symbols = [doc.symbols; symbol];
            end
            unique_symbols(colons) = {':'};
            for p = 1:length(unique_symbols)
              ind = symbols_in_text == unique_symbols{p};
              doc.symbols.count(p) = nnz(ind);
            end
            num = sum(doc.symbols.count);
            doc.symbols.Properties.CustomProperties.num_symbols = num;
          case "letters"
            % parse the upper case letters
            if isstring(doc.text)
              new_text = char(doc.text);
            else
              new_text = doc.text;
            end
            ind = isupper(new_text);
            uppers_in_text = new_text(ind);
            unique_uppers = unique(uppers_in_text, 'stable');
            for p = 1:length(unique_uppers)
              ind = uppers_in_text == unique_uppers(p);
              doc.letters.upper(unique_uppers(p)) = nnz(ind);
            end
            % parse the lower case letters
            ind = islower(new_text);
            lowers_in_text = new_text(ind);
            unique_lowers = unique(lowers_in_text, 'stable');
            for p = 1:length(unique_lowers)
              ind = lowers_in_text == unique_lowers(p);
              doc.letters.lower(upper(unique_lowers(p))) = nnz(ind);
            end
            num = sum(doc.letters{:,:});
            doc.letters.Properties.CustomProperties.num_letters = num;
        end
      end
\end{verbatim}
\begin{verbatim}
    end
    % ==
    function [w s f] = find(doc, target, options)
\end{verbatim}
\begin{verbatim}
      % ------------------------
      % - returns an array
      %   containg all of the
      %   words in the text
      %   that match a
      %   certain target word
      % - also returns the start
      %   and ending indices of
      %   these words
      % ------------------------
\end{verbatim}


\subsection*{check the input arguments}

\begin{par}
check the argument classes
\end{par} \vspace{1em}
\begin{verbatim}
      arguments
        doc;
        target (1,:) {mustBeText};
        options.Mode (1,:) ...
        {mustBeText, ...
         mustBeMemberi(options.Mode, ["matches" "contains"])} = "matches";
        options.IgnoreCase (1,:) ...
        {mustBeNumericOrLogical, mustBeNonempty} = false;
      end
      target = string(target);
      Mode = string(lower(options.Mode));
      IgnoreCase = options.IgnoreCase;
      % check the argument dimensions
      [~, Mode IgnoreCase] = scalar_expand(target, Mode, IgnoreCase);
      if ~isequallen(target, Mode, IgnoreCase)
        str = stack('''Mode'', and ''IgnoreCase''', ...
                    'must have the same length', ...
                    'as ''target'', or be scalars');
        error(str);
      end
\end{verbatim}


\subsection*{case for non-scalar documents}

\begin{verbatim}
      if isempty(doc)
        [w s f] = deal({}, [], []);
        return;
      elseif ~isscalar(doc)
        [w s f] = deal(cell(size(doc)));
        Args = {'Mode' Mode 'IgnoreCase' IgnoreCase};
        for k = 1:numel(doc)
          [w{k} s{k} f{k}] = doc(k).find(target, Args{:});
        end
        return;
      end
\end{verbatim}


\subsection*{compute the word information from the table}

\begin{verbatim}
      Doc = copy(doc);
      if isempty(Doc.words)
        Doc.parse('words');
      end
      if isstring(Doc.text)
        [w s f] = deal(string.empty, [], []);
        wt = string(Doc.words.word);
      else
        [w s f] = deal({}, [], []);
        wt = Doc.words.word;
      end
      st = Doc.words.Properties.CustomProperties.s;
      ft = Doc.words.Properties.CustomProperties.f;
\end{verbatim}


\subsection*{find the words in the text matching the target word}

\begin{verbatim}
      for k = 1:length(target)
        if Mode(k) == "matches"
          ind = matches(wt, target(k), 'IgnoreCase', IgnoreCase(k));
        else
          ind = contains(wt, target(k), 'IgnoreCase', IgnoreCase(k));
        end
        ind = ind & ~ismember(wt, w);
        if any(ind)
          w = [w; wt(ind)];
          s = [s; st(ind)];
          f = [f; ft(ind)];
        end
      end
\end{verbatim}


\subsection*{modify the outputs to a more convenient type}

\begin{verbatim}
      if iscellscalar(w)
        w = w{1};
        s = s{1};
        f = f{1};
      end
\end{verbatim}
\begin{verbatim}
    end
    % ==
    function [w s f] = find_bad_words(doc, options)
\end{verbatim}
\begin{verbatim}
      % ------------------------
      % - returns an array
      %   containg all of the
      %   bad words in the text
      % - also returns the start
      %   and ending indices of
      %   these words
      % ------------------------
\end{verbatim}


\subsection*{check the input arguments}

\begin{verbatim}
      arguments
        doc;
        options.Mode ...
        {mustBeTextScalar, ...
         mustBeMemberi(options.Mode, ["matches" "contains"])} = "contains";
        options.IgnoreCase (1,1) logical = true;
      end
      Args = namedargs2cell(options);
\end{verbatim}


\subsection*{find the bad words}

\begin{verbatim}
      [w s f] = doc.find(document.bad_words, Args{:});
\end{verbatim}
\begin{verbatim}
    end
    % ==
    function bool = has_bad_words(doc)
      % --------------------------
      % - returns true if the text
      %   contains bad words
      % --------------------------
      IgnoreCase = {'IgnoreCase' true};
      bool = false(size(doc));
      for k = 1:numel(doc)
        bool(k) = contains(doc(k).text, document.bad_words, IgnoreCase{:});
      end
    end
    % ==
    function varargout = wordcloud(doc, varargin)
      % ----------------------
      % - plots the word cloud
      %   of the document
      % ----------------------
      if ~isscalar(doc)
        error('the document must be a scalar');
      end
      Doc = copy(doc);
      if isempty(Doc.words)
        Doc.parse('words');
      end
      Doc.words = addvars(Doc.words, Doc.words.word, 'Before', 'count');
      varargin = [{Doc.words 'Var1' 'count'} varargin];
      [varargout{1:nargout}] = wordcloud(varargin{:});
    end
    % ==
  end
  % ==
end
\end{verbatim}

        \color{lightgray} \begin{verbatim}
ans = 

\end{verbatim} \color{black}
    


\end{document}

